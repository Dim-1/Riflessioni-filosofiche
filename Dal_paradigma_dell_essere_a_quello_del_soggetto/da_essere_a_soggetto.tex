\documentclass[a4paper,12pt,oneside]{article}%book o scrbook, da provare; twoside per libro; b5 per libro piccolo
\usepackage[T1]{fontenc}
\usepackage[utf8]{inputenc}
\usepackage[italian]{babel}
%\usepackage{layaureo}
%\usepackage[eulerchapternumbers,beramono,pdfspacing]{classicthesis}
%\usepackage{arsclassica}

\begin{document}
	\author{Sandro Della Maggiore}
	\title{Il passaggio dalla filosofia dell'essere alla filosofia del soggetto}
	\date{Aprile 2024}
	
	\maketitle

Da Cartesio in avanti il soggetto è l'elemento indubitabile e inaggirabile, che non può essere evitato, in quanto la sua messa in discussione lo richiama sempre in causa. L'argomento cartesiano "Cogito, ergo sum" è detto confutativo, ovvero la messa in discussione del soggetto lo presuppone, quindi la confutazione stessa decade; meccanismo simile fu applicato da Aristotele nel IV libro della metafisica riguardo il principio di non contraddizione. Questo argomento è deto anche trascendentale, ovvero intrascindibile, non aggirabile, che ritroviamo sempre alle spalle.

In epoca moderna quindi il paradigma della filosofia passa dalla centralità dell'essere e della sostanza al paradigma del soggetto (coscienza, spirito). Il paradigma ontologico degli antichi poneva come base l'essere delle cose, cioè l'indubitabilità dell'esistenza delle cose: la base del mondo è la sostanza (ousìa), e la sostanzialità è la proprietà che hanno le cose di esistere. Che un ente sia sostanza, da Platone in avanti, significa che ha in se stesso le ragioni della sua esistenza, che è fondato su se stesso: ousìa struttura fondamentale delle cose.

Per Platone, la ragion d'essere defli enti non sta in questo mondo, ma nell'iperuranio, nel mondo metafisico: ousìa sono le idee, sono oltre il sensibile e sono la ragion d'essere del mondo. Per Aristotele le sostanze sono nel mondo fisico, ovvero gli enti fisici hanno in loro stessi la ragione della loro esistenza. Aristotele però è un critico del naturalismo: ciò che consente ad un certo ente di essere, di avere sostanza, è la "forma" (eidos); la forma è la vera sostanza, che nel mondo fisico organizza la grezza materia per conferirgli sostanzialità. Ad allontanare ulteriormente Aristotele dal naturalismo è la giustificazione del divenire e del mutamento delle sostanze con una causa metafisica, il motore immobile, il pensiero di pensiero. La posizione aristotelica quindi vede la ragione di esistenza delle cose nelle cose stesse, mentre nell'aldilà metafisico si trova la causa del movimento e del mutamento.

Per gli antichi l'essere è indubitabile, non c'è ragione di dubitare che le cose siano sostanza (cioè che hanno la proprietà di esistere), e la tesi della metafisica antica è che il fondamento del mondo fisico non sta in questo mondo: questo fondamento inteliggibile è assoluto, cioè indubitabile, necessario e incotrovertibile. \textbf{La verità dell'aldiquà sta nell'aldilà.}


	
\end{document}
