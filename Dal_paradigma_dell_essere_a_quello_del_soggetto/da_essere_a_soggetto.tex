\documentclass[a4paper,12pt,oneside]{article}%book o scrbook, da provare; twoside per libro; b5 per libro piccolo
\usepackage[T1]{fontenc}
\usepackage[utf8]{inputenc}
\usepackage[italian]{babel}
%\usepackage{layaureo}
%\usepackage[eulerchapternumbers,beramono,pdfspacing]{classicthesis}
%\usepackage{arsclassica}

\begin{document}
	\author{Sandro Della Maggiore}
	\title{Il passaggio dalla filosofia dell'essere alla filosofia del soggetto}
	\date{Aprile 2024}
	
	\maketitle

Da Cartesio in avanti il soggetto è l'elemento indubitabile e inaggirabile, che non può essere evitato, in quanto la sua messa in discussione lo richiama sempre in causa. L'argomento cartesiano "Cogito, ergo sum" è detto confutativo, ovvero la messa in discussione del soggetto lo presuppone, quindi la confutazione stessa decade; meccanismo simile fu applicato da Aristotele nel IV libro della metafisica riguardo il principio di non contraddizione. Questo argomento è deto anche trascendentale, ovvero intrascindibile, non aggirabile, che ritroviamo sempre alle spalle.

In epoca moderna quindi il paradigma della filosofia passa dalla centralità dell'essere e della sostanza (ontologico) al paradigma del soggetto (coscienza, spirito).
	
\end{document}
