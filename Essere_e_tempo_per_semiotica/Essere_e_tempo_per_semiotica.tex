\documentclass[a4paper,12pt,oneside]{article}%book o scrbook, da provare; twoside per libro; b5 per libro piccolo
\usepackage[T1]{fontenc}
\usepackage[utf8]{inputenc}
\usepackage[italian]{babel}
%\usepackage{layaureo}
%\usepackage[eulerchapternumbers,beramono,pdfspacing]{classicthesis}
%\usepackage{arsclassica}

\begin{document}
	\author{Sandro Della Maggiore}
	\title{Essere e tempo per la semiotica}
	\date{Luglio 2024}
	
	\maketitle


"Essere e Tempo" (Sein und Zeit) è un’opera del 1927 del filosofo tedesco Martin Heidegger. Divisa in tre sezioni, è un’opera incompiuta, infatti l’ultima parte non è mai stata realizzata. In Essere e Tempo Heidegger affronta il problema dell’\textit{essere}, o meglio, del senso dell'\textit{essere}; questo problema è stato eluso per millenni da tutto il pensiero filosofico occidentale, per vari motivi, che lui controbatte:

\begin{enumerate}
	\item L'\textit{essere} è il concetto più generale, perciò vuoto, senza contenuti da esplicare; per H. ciò significa che l'\textit{essere}, nella sua generalità, è il concetto più oscuro, e va quindi chiarito.
	\item Non possiamo affermare "l'\textit{essere} è ..." senza usare il verbo "essere", quindi l'\textit{essere} è indefinibile; per H. l'impossibilità di definire l'\textit{essere} indica che esso non è un'ente, quindi cosa è?
	\item \textit{Essere} è un concetto ovvio, dove già ci intendiamo, non vale la pena soffermarcisi; per H. il fatto di \textit{pre-comprendere} l'\textit{essere} richiede ancora più fortemente la comprensione del suo senso, di capire cosa sappiamo senza saperlo.
\end{enumerate}

La domanda che ci poniamo dunque è questa: "Cosa significa che l'ente \underline{è} ed \underline{è} così?

Vi è un cercato, l'\textit{essere}, o meglio il suo senso, e il cercante, noi uomini che pre-comprendiamo l'\textit{essere}, ponendo il problema di questa pre-comprensione stessa. L'\textit{essere} è ciò che determina l'ente in quanto tale: se fosse stato a sua volta un ente che desse modo agli enti di esistere, sarebbe Dio o qualcosa di metafisico; H. rifiuta questa soluzione, egli vuole cercare l'\textit{essere} nell'ente, come mai è stato fatto in precedenza. Al che sorge la domanda: in quale ente va cercato l'\textit{essere}? In quello che già lo pre-comprende, nel cercante, l'uomo, che da sempre si rapporta con l'\textit{essere}.

H. designa noi cercanti con il termine di \textit{esser-ci} (\textit{das sein}, tradotto da altri interpreti anche come esser-qui): caratteristica fondamentale dell’uomo è la sua esistenza, che si realizza dentro a un certo tempo e un certo spazio. L’uomo, dunque, è tale perché esiste. Questa essenza dell'esser-ci, l'esistenza dell'uomo, è di tipo interpretativo, ermeneutico, ovvero noi come enti dobbiamo sempre comprenderci, costruirci prospettive di vita, vista l'indefinibilità dell'\textit{essere} citata precedentemente. La comprensione della propria esistenza può avvenire in due modi, che distinguano due stili di vita:

\begin{itemize}
	\item Vita inautentica, mediante pre-comprensione non concettuale dell'\textit{essere} dell'esser-ci.
	\item Vita autentica, mediante comprensione autentica dell'esistenza, resa possibile dal cammino intrapreso nell'opera di H..
\end{itemize} 
	
Nell'introduzione di "Essere e Tempo" H. puntualizza che la comprensione dell'\textit{essere} dell'esser-ci (cioè dell'esistenza) concerne co-originariamente la comprensione del mondo e la comprensione dell'\textit{essere} dell'ente (cioè le cose che ci circondano) nel mondo. L'esistenza dell'uomo è nel mondo, è relazione con il mondo, e l'esser-ci non \textit{ci-è} senza mondo. L'\textit{in-essere}, cioè l'\textit{essere}-nel-mondo, è un \textit{esistenziale} secondo H., è costitutivo dell'uomo, in quanto l'uomo è ente che sempre si comprende e interpreta come legato all'\textit{essere} dell'ente che incontra all'interno del proprio mondo\footnote{Esistenziale in Heidegger corrisponde al termine trascendentale, usato da Kant in poi. Trascendentale indica la condizione di possibilità, cioè la condizione a cui devono sottostare le possibilità che si presentano: per esempio in Kant l'intelletto trascendentale cataloga i dati attraverso le categorie, che rappresentano la condizione attraverso cui possiamo apprendere. Per essere precisi, in H. l'esistenziale è una condizione condizionata di possibilità, in quanto l'esser-ci ha pre-comprensione come caratteristica attraverso cui comprende il mondo, caratteristica che a sua volta è condizionata dal \textit{qui}, dalla cultura del luogo dove siamo al mondo.}. Ontologicamente, comprendiamo chi siamo incontrando altri enti che comprendiamo non essere noi.

Il mondo è l'orizzonte in cui dobbiamo operare: questo orizzonte cambia da individuo a individuo (o forse meglio dire tra macrogruppi di individui), data la natura interpretativa dell'esistenza dell'uomo nel mondo.

L’esser-ci è accerchiato da cose a cui dare un significato utile alla realizzazione dei suoi progetti: perciò l’esistenza dell’esser-ci è caratterizzata dalla possibilità; l’uomo ha davanti a sé indefinite possibilità da realizzare, che si traducono nella possibilità di progettare. Questo libertà dell'uomo è una conseguenza del rapporto con l'\textit{essere} che abbiamo da sempre, rapporto ermeneutico, bene ricordarlo.

Adesso il problema si sposta sulla \textit{mondità} del mondo, l'\textit{essere} mondo del mondo. Non si deve procedere:

\begin{enumerate}
	\item Enumerando e descrivendo gli enti del mondo, rimanendo sul piano \textit{ontico} (relativo all'ente) e non ontologico (relativo all'\textit{essere}); qualsiasi descrizione presuppone la \textit{mondità}.
	\item Svelando l'\textit{essere} dell'ente presente nel mondo, ovvero la natura, perché così facendo presupponiamo la \textit{mondità} della natura.
\end{enumerate}

Poiché l'essere-nel-mondo è un esistenziale, un carattere costitutivo dell'esistenza umana, la mondità va cercata nell'uomo; dobbiamo condurre un'"analitica dell'esser-ci" per indagare il mondo \footnote{Nota personale: si ricade in qualche tipo di idealismo? Il soggetto proietta le sue leggi sul mondo-oggetto o addirittura pone il mondo come contrapposizione al soggetto-uomo.}.
	
Il mondo più prossimo all'esser-ci nella sua quotidianità è il mondo-ambiente: l'uomo si comprende inanzi tutto attraverso "un commercio con il mondo e con gli enti intramondani", non mediante il conoscere percettivo, ma attraverso il "prendersi cura (da non intendersi in senso morale) usante e maneggiante" dell'ente (strumento del e nel mondo); usando e maneggiando oggetti, l'essere umano comprende cosa egli non è, e l'esistenza viene interpretata a partire dall'ente nell'ambiente che ci circonda, che ha il suo senso d'\textit{essere} in quanto mezzo e strumento, tanto da affermare che l'\textit{utilizzabilità} è il modo d'\textit{essere} del mezzo. Importante notare che l'esistenza umana è compresa non attraverso la contemplazione della natura-oggetto (percezione), che è legata al momento presente, bensì attraverso l'interazione e l'uso degli strumenti della natura o da essa ricavati, che danno una prospettiva temporale all'\textit{essere} dell'esserci; manipolando e progettando, l'uomo trova il suo significato anche nel passato e nel futuro.


Ogni mezzo dell'uomo, non è mezzo isolato, bensì appartiene alla totalità dei mezzi, con cui è in relazione: ogni mezzo è "qualcosa per ..." che rinvia ad un ulteriore "qualcosa per ...", eccetera. Il "per" contiene implicitamente sempre un rimandare di qualcosa a qualcos'altro. Ogni commercio con un mezzo sottostà alla molteplicità dei rimandi costitutivi del "per", ogni volta che uno strumento è usato rimanda ad un altro e così via attraverso le infinite relazioni di tutti gli enti del mondo tra loro: \textit{la visione ambientale preveggente} è la visione connessa a questo processo, cioè, ogni volta che l'uomo progetta, mette in relazione le cose del mondo tra loro (esempio: una lavagna assume un certo significato se è posta vicino ad una cattedra, perché la presenza della cattedra rimanda all’idea che ci troviamo in un’aula), in maniera inconscia, senza consapevolezza e tematizzazione, attraverso pre-comprensione (in pratica attraverso la visione ambientale preveggente ci muoviamo nel mondo come sempre l'abbiamo conosciuto, grazie a tradizioni, educazione, linguaggio).

Collegato a questo tema del rimando vi è poi quello fondamentale, già accennato, della pre-comprensione. Quando siamo in un atteggiamento di comprensione del mondo, questa comprensione non arriva mai dal nulla, ma parte sempre da una rete di significati già presenti in noi: questi significati derivano dalla rete di rimandi che abbiamo descritto (la visione ambientale preveggente, nel senso che l’ambiente condiziona la nostra comprensione delle cose), ma anche da rimandi più generali, come la nostra storia personale, la nostra cultura, e via dicendo. Di tutti questi strumenti di pre-comprensione il più importante è il linguaggio, in quanto è strumento di comunicazione di base con cui pensiamo e diciamo il mondo. A cascata,  il tema della pre-comprensione ci porta alla questione del \textit{circolo ermeneutico}. Da quello che abbiamo visto, conoscere il mondo non significa conoscerlo ex novo, ma interpretarlo. Questo significa che per una comprensione più adeguata dobbiamo scavare nell’interpretazione del mondo, e ogni passaggio di questa interpretazione arricchisce ulteriormente la comprensione. Si innesca quindi una circolarità di significati, un circolo ermeneutico appunto. Usando un’immagine metaforica, potremmo dire che la comprensione non è tanto un andare avanti, ma un andare indietro, ovvero ripercorrere la catena di significati che è alle spalle di una situazione attuale.

Prima di tornare al nostro cammino che ci condurrà alla semiotica, vale la pena sviluppare degli argomenti che riguardano il progettare dell'uomo come suo rapporto verso il mondo: con la dimensione della cura per l'ente, Heidegger analizza come l’esser-ci si pone in relazione alla dimensione temporale.
Qui entra in gioco il tema della differenza fra vita autentica e vita inautentica.
La vita autentica è caratterizzata dall'\textit{appartenere}, nasce da progetti che ci appartengono, che sono propriamente nostri; viceversa, la vita inautentica è legata ad una dimensione in cui non sviluppiamo progetti che sentiamo come nostri.
Per comprendere questa distinzione introduciamo il concetto di mondo del \textit{si} (\underline{si} dice, \underline{si} fa, ...). Questa dimensione è quella da cui tutti, necessariamente, passiamo. Il problema è che rischiamo di rimanerci incastrati e vivere nelle tre trappole che il mondo del si ci pone: la chiacchiera, ovvero limitarci a pensare e dire le cose che generalmente si pensano e si dicono, reputandole vere in quanto di senso comune; la curiosità, ovvero l’interessarci della vita altrui rimanendo nella superficie dell’apparenza visibile; l’equivoco, ovvero pensare che quello che emerge dalla chiacchiera e dalla curiosità rappresenti realmente la verità.
Rimanendo dentro a questa dimensione entriamo nella vita inautentica, in quanto rinunciamo a scavare la verità e finiamo per vivere una vita che non ci appartiene, una vita di conformismo. Subiamo un processo di deiezione, ovvero diventiamo una cosa fra le cose, una semplice presenza. Rinunciamo alla nostra esistenza autentica, ovvero a produrre progetti che ci appartengono.
L’alternativa alla vita inautentica nasce se ci poniamo come esseri-per-la-morte. Ovvero: prendiamo coscienza della dimensione più profonda della morte. La morte è la possibilità più paradossale, in quanto il suo giungere pone fine a tutte le altre possibilità esistenziali. Questo significa anche che la morte è l’unica possibilità necessaria, che cancella l’esistenza. Eppure, l’orizzonte della morte può essere decisivo per costruire una vita autentica: se accettiamo la necessità della morte, e quindi la necessità del nulla, accettiamo la nostra esistenza per quello che è. Invece di fuggire la morte non pensandoci, la trasformiamo in decisione anticipatrice. Questo significa che possiamo riempire il tempo di significato, possiamo trasformare il tempo in istante, per usare un’immagine di Nietsche, ovvero in un attimo denso di significato. Attraverso questa operazione recuperiamo appieno il nostro rapporto col tempo: la deiezione della vita inautentica ci spinge in una sorta di eterno presente, la decisione anticipatrice fa calare nel presente dei nostri progetti lo sguardo sul futuro.

Le esperienze tramite cui illuminiamo il mondo, i momenti in cui emerge il carattere di rimando proprio di ogni utilizzabile in quanto mezzo, avvengono quando scopriamo l'inutilizzabilità di un oggetto utilizzabile, permettendoci di andare oltre la pre-comprensione che ci fa assumere tutto per scontato. Le esperiende dove pesdiamo l'utilizzabilità di un ente sono tre:

\begin{itemize}
	\item sorpresa, quando l'oggetto non è idoneo;
	\item importunità, quando l'oggetto è mancante;
	\item impertinenza, quando l'oggetto ostacola il nostro progetto.
\end{itemize}

Con l'apparire di almeno uno dei tre imprevisti, il mondo compare ai nostri occhi, si stacca dallo sfondo "già costantemente visto sin dal principio nel corso della visione ambientale preveggente, visto in maniera non tematica (cioè pre-compreso ma non compreso). Questa emersione del mondo dovuta all'errore può permettere la tematizzazione del mondo stesso e la sua comprensione intesa come totalità di tutti i rimandi. \textbf{Ogni rimando è un segno, cioè un mezzo il cui specifico carattere di mezzo è quello di indicare}. 

Essendo ogni ente usato e interpretato (\textit{curato}) dall'uomo, ogni oggetto è un segno che rimanda ad un'interpretazione o ad un significato, che nella visione ambientale preveggente fa emergere un complesso di mezzi così da far annunciare la conformità al mondo propria dell'utilizzabile. Ogni segno annuncia "ciò che sta venendo" a cui non eravamo preparati perché indaffarati in altro (esempio: la lavagna nera ci annuncia una classe, una cattedra e una scuola), indicano dove si vive.
L'essere dell'ente utilizzabile ha la struttura del rimando, ogni oggetto ha in sé il carattere di rimandare e di essere rimandato. 

"Ogni ente ha con sé, presso qualcosa, il suo appagamento": secondo H. l'utilizzabile è caratterizzato dall'\textit{appagatività}, che è l'\textit{essere} dell'ente intramondano, ciò a cui esso è rimesso, il suo scopo di utilizzo (esempio: il martello è appagato quando martella; la lavagna quando è scritta e sta in una classe). Così come ogni utilizzabile rimanda a qualcosa, altrettanto l'appagatività è inserita in una catena: martellare serve per costruire, che serve per edificare, che serve per abitare. La catena mette capo, infine, alla \textit{totalità delle appagatività}, distinta come:

\begin{itemize}
	\item L'\textit{in-vista-di-cui}: l'esser-ci è ciò verso cui si muove la catena delle appagatività, verso cui l'utilizzabile si manifesta nella sua catena di rimandi; in altre parole tutti gli strumenti hanno la loro utilizzabilità (appagatività) stabilita dall'interpretazione che l'uomo gli attribuisce, mettendoli in relazione l'uno all'altro in base ai suoi progetti.
	\item Il \textit{presso-che}: l'appagatività è già sempre compresa nel mondo pre-compreso da sempre dall'uomo.
\end{itemize} 

Il mondo è l'orizzonte ermeneutico dell'uomo: l'esserci è sempre un'interpretazione di sé, del suo essere nel mondo. L'appagatività di ogni utilizzabile è resa possibile dall'aver sempre pre-compreso il mondo (carattere esistenziale dell'esser-ci): in altre parole, far parte di una determinata cultura e tradizione, assegna a tutto quello che ci circonda un significato tipico di quella certa cultura.

Possiamo concludere che la comprensione (interpretazione) del mondo è la mondità del mondo, che rivela la catena del rimandare, tutti i suoi rapporti di appagatività (utilizzabilità). Se per significare intendiamo il carattere di rimandare dei rapporti, la totalità dell'appagatività è la significatività del mondo.

L'esser-ci significa (rimanda) a se stesso che ha da conoscere il suo \textit{essere} e il suo poter \textit{essere} a partire dal suo \textit{essere} nel mondo: ogni uomo si distingue da un altro in base a come interpreta il mondo, a come vi si interfaccia e vi si muove. Avendo però ogni uomo una pre-comprensione del mondo (esempio: la cultura e la popolazione in cui nasciamo), esso si ritrova già in una certa struttura del mondo, espressa dalla significatività (totalità dei rapporti del significare).

L'esser-ci è sempre rinviato ad un mondo che gli viene incontro, che è strutturato come significatività (totalità dei rapporti di appagatività/ utilizzabilità). Il mondo (cui apparteniamo) porta con sè la possibilità di essere compreso da parte dell'esser-ci, trovando i significati (di tutti gli enti, le loro relazioni, il loro utilizzo), i quali, a loro volta, fondano la possibilità della parola e del linguaggio.

La comprensione della significatività avviene da parte dell'esser-ci: il "\textit{ci} (il qui)" significa apertura esistenziale, perché da parte dell'uomo l'\textit{essere}-nel-mondo è un esistenziale (una condizione a monte di tutto, l'esistenza è trascendentale), per cui ha proprio il carattere di non-chiusura. L'\textit{essere qui} dell'esser-ci si comprende a partire dal \textit{là} (l'utilizzabile), ovvero l'uomo deve interpretarsi nel suo \textit{essere}-nel-mondo: questa originaria comprensione (quale si manifesta nella pre-comprensione) assegna il \textit{qui} all'esser-ci e il \textit{là} al mondo.

Alla comprensione è coessenziale la \textit{situazione emotiva}: "l'affettività propria della situazione emotiva è un elemento esistenziale costitutivo dell'apertura dell'esser-ci al mondo"; solo per tale apertura i sensi sentono e subiscono affezioni o sensazioni, quindi il sentire è originariamente emotivo e non contemplativo-conoscitivo. Questo \textit{essere} emotivamente aperto consiste in ciò: che l'esser-ci è un \textit{esser-gettato} di questo ente nel suo \textit{ci} (l'uomo è catapultato nel mondo), gettato nel senso \textit{che c'è e che ha da essere}, restando però nascosto il \textit{donde (perché) e il dove}. Possiamo avere varie credenze sul perché dell'\textit{esser-gettato}: per esempio appellarsi ad una fede religiosa, affidarsi alla scienza. Queste interpretazioni sul piano ontico vengono dopo quel \textit{autosentimento situazionale} caratteristico dell'uomo; esse sono risposte che i creiamo per dare significato alla situazione emotiva, di cui non siamo consapevoli.

Nella quotidianità l'esser-ci è nell'atteggiamento \textit{dell'evasione e della fuga}, ricorrendo a volontà e sapere per padroneggiare le proprie emozioni \footnote{Nietzsche disse che la volontà di sapere sorge per sopprimere la paura, per darsi interpretazioni del mondo sempre più sopportabili e piacevoli.}. Grazie alle emozioni, tra cui la paura, l'uomo interpreta il mondo.

H. si sofferma sull'analisi della paura: essa ha un suo \textit{davanti-a-che}, cioè un ente che si incontra nel mondo e che genera paura. Questo ente può essere: un utilizzabile, una semplice presenza o un altro esser-ci (cioè un \textit{con-esser-ci}). L'appagatività di un ente che viene incontro minaccioso è la \textit{dannosità}. L'esser-ci, in quanto \textit{essere-nel-mondo} (quindi proiettato verso l'esterno), è spaurito, ha da sopportare il peso del suo \textit{esser qui}, e nell'aver da \textit{essere} ne va del suo \textit{essere}: è costitutivamente di fronte all'enigma dell'\textit{esser-gettato}, in cui donde e dove  restano oscuri.

Questo \textit{essere} dell'esser-ci è legato al destino dell'ente intramondano che l'uomo incontra coessenzialmente maneggiando e usando, cioè interpretando, o meglio, pre-comprendendo. La comprensione è sempre emotivamente tonalizzata, è un originario volgersi dell'uomo alla sua natura da sempre interpretativa, e non è sapere concettuale o spiegazione. Il maneggiare l'ente da parte dell'uomo denota la sua temporalità, mentre la vecchia metafisica pretendeva di cogliere l'ente come immutabile, cogliendo solo il momento presente dell'ente anche quando ammette il divenire, in quanto pretende di fermare il concetto nel tempo. Un ente non lo si può contemplare, perché nel frattempo passa e muta, invece lo si può capire, apprendere e interpretare maneggiare e usandolo.

L'esser-ci, in quanto comprende e interpreta, è un poter \textit{essere}, è possibilità, è un \textit{esser-possibile} gettato nel mondo: ha da \textit{essere} il suo \textit{qui} interamente (vita inautentica) oppure smarrirsi dal suo \textit{qui}  per poi ritrovarsi (vita autentica: andare oltre la propria cultura, diventare ciò che si è).
Le nostre possibilità sono interamente nella totalità di appagatività degli utilizzabili. Rivolgersi a questo tutto delle possibilità significa \textit{progettare}: progetto è la struttura profonda della comprensione, è interazione attiva con il mondo, ben oltre la pre-comprensione che ci è data da cultura e tradizioni, è diventare ciò che siamo e non un semplice duplicato culturale. Nel linguaggio di H., l'interpretazione è l'elaborazione delle possibilità progettate nella comprensione: avendo pre-compreso, cioè avendo una visione ambientale preveggente, possiamo successivamente interpretare, cioè elaborare la pre-comprensione per ricavare comprensione.

L'interpretazione, prendendosi cura dell'utilizzabile, rende esplicita le sue relazioni con il modo circostante (il suo \textit{per}), esplica la comprensione; non necessariamente nella forma della predicazione, cioè del giudizio e del linguaggio. L'interpretazione avviene sotto due condizioni:

\begin{enumerate}
	\item ha luogo da una totalità di appagatività pre-compresa, cioè avviene quando siamo completamente calati dentro una cultura (non esiste uomo se non dentro ad altri gruppi umani);
	\item lo svelamento e l'appropriazione del compreso si realizza sotto la guida di una prospettiva che stabilisce la direzione in cui il compreso deve essere interpretato.
\end{enumerate}

Perciò l'interpretazione non è mai apprendimento neutrale di qualcosa di dato (sarebbe il sogno dello storicismo e del positivismo/scientismo), e il dato immediato è solo un'assunzione dell'interpretante.

Il mondo acquista un senso che è implicito nella pre-comprensione, facendosi esplicito nell'interpretazione: vanno così a coincidere la mondità del mondo pre-compreso con la significatività, così che l'orizzonte del mondo coincide con quello ermeneutico. Inoltre, ogni interpretazione deve in qualche modo già aver interpretato per poter interpretare. Eccoci alla questione del \textit{circolo ermeneutico}: è un circolo vizioso, in quanto si presuppone ciò che si vorrebbe interpretare, poiché ogni interpretazione parte da una pre-comprensione di ciò che deve interpretare. Le scienze moderne rifiutano il circolo, fallendo perciò il chiarimento della comprensione originaria. Si deve stare nel circolo nel giusto modo, perché interpretare caratterizza la natura umana: non dobbiamo farci influenzare da pregiudizi, opinioni comuni ("si dice", "si fa"), piuttosto dobbiamo far emergere spontaneamente la nostra prospettiva, come cosa che ci caratterizza nel nostro \textit{esser-gettati}, ovvero destinati.

La significatività generale dell'\textit{esser-nel-mondo} rende possibile l'articolazione dei significati, il loro legarsi: su questa interpretazione si fonda la possibilità del linguaggio e della parola. La comunicazione realizza la compartecipazione della situazione emotiva comune della comprensione del \textit{con.essere} (intersoggettività). Il linguaggio  è l'espressione del discorso (logos), a sua volta esistenziale cooriginario alla situazione emotiva e alla comprensione, in quanto articola la comprensibilità, stando alla base dell'interpretazione e dell'asserzione. Il logos svela l'interpretazione, articola la comprensibilità dell'\textit{esser qui} dell'esser-ci nei significati caratterizzanti l'\textit{esser qui} dell'uomo. D'altra parte il linguaggio diviene "parola (de Saucerre)", cioè un \textit{essere} mondano utilizzabile. H. si chiede se il linguaggio ha il modo d'essere dell'utilizzabile intramondano ("parole") o il modo d'\textit{essere} dell'esser-ci: la linguistica lo ignora.





\end{document}