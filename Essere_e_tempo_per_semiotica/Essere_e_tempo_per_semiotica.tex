\documentclass[a4paper,12pt,oneside]{article}%book o scrbook, da provare; twoside per libro; b5 per libro piccolo
\usepackage[T1]{fontenc}
\usepackage[utf8]{inputenc}
\usepackage[italian]{babel}
%\usepackage{layaureo}
%\usepackage[eulerchapternumbers,beramono,pdfspacing]{classicthesis}
%\usepackage{arsclassica}

\begin{document}
	\author{Sandro Della Maggiore}
	\title{Essere e tempo per la semiotica}
	\date{Luglio 2024}
	
	\maketitle


"Essere e Tempo" (Sein und Zeit) è un’opera del 1927 del filosofo tedesco Martin Heidegger. Divisa in tre sezioni, è un’opera incompiuta, intatti l’ultima parte non è mai stata realizzata. In Essere e Tempo Heidegger affronta il problema dell’\textit{essere}, o meglio, del senso dell'\textit{essere}; questo problema è stato eluso per millenni da tutto il pensiero filosofico occidentale, per vari motivi, che lui controbatte:

\begin{enumerate}
	\item L'\textit{essere} è il concetto più generale, perciò vuoto, senza contenuti da esplicare; per H. ciò significa che l'\textit{essere}, nella sua generalità, è il concetto più oscuro, e va quindi chiarito.
	\item Non possiamo affermare "l'\textit{essere} è ..." senza usare il verbo "\textit{essere}", quindi l'\textit{essere} è indefinibile; per H. l'impossibilità di definire l'\textit{essere} indica che esso non è un'ente, quindi cosa è?
	\item \textit{Essere} è un concetto ovvio, dove già ci intendiamo, non vale la pena soffermarcisi; per H. il fatto di \textit{pre-comprendere} l'\textit{essere} richiede ancora più fortemente la comprensione del suo senso, di capire cosa sappiamo senza saperlo.
\end{enumerate}

La domanda che ci poniamo dunque è questa: "Cosa significa che l'ente \underline{è} ed \underline{è} così?

Vi è un cercato, l'\textit{essere}, o meglio il suo senso, e il cercante, noi uomini che pre-comprendiamo l'\textit{essere}, ponendo il problema di questa pre-comprensione stessa. L'\textit{essere} è ciò che determina l'ente in quanto tale: se fosse stato un ente che desse modo agli enti di esistere, sarebbe Dio o qualcosa di metafisico; H. rifiuta questa soluzione, egli vuole cercare l'\textit{essere} nell'ente, come mai è stato fatto in precedenza. Al che sorge la domanda: in quale ente va cercato l'\textit{essere}? In quello che già lo pre-comprende, nel cercante, l'uomo, che da sempre si rapporta con l'\textit{essere}.

H. designa noi cercanti con il termine di \textit{esser-ci} (\textit{das sein}, tradotto da altri interpreti anche come esser-qui): caratteristica fondamentale dell’uomo è la sua esistenza, che si realizza dentro a un certo tempo e un certo spazio. L’uomo, dunque, è tale perché esiste. Questa essenza dell'esser-ci, l'esistenza dell'uomo, è di tipo interpretativo, ermeneutico, ovvero noi come enti dobbiamo sempre comprenderci, costruirci prospettive di vita, vista l'indefinibilità dell'\textit{essere} citata precedentemente. La comprensione della propria esistenza può avvenire in due modi, che distinguano due stili di vita:

\begin{itemize}
	\item Vita inautentica, mediante pre-comprensione non concettuale dell'\textit{essere} dell'esser-ci.
	\item Vita autentica, mediante comprensione autentica dell'esistenza, resa possibile dal cammino intrapreso nell'opera di H..
\end{itemize} 
	
Nell'introduzione di "Essere e Tempo" H. puntualizza che la comprensione dell'\textit{essere} dell'esser-ci (cioè dell'esistenza) concerne co-originariamente la comprensione del mondo e la comprensione dell'\textit{essere} dell'ente (cioè le cose che ci circondano) nel mondo. L'esistenza dell'uomo è nel mondo, è relazione con il mondo, e l'esser-ci non \textit{ci-è} senza mondo. L'\textit{in-essere}, cioè l'\textit{essere}-nel-mondo, è un \textit{esistenziale} secondo H., è costitutivo dell'uomo, in quanto l'uomo è ente che sempre si comprende e interpreta come legato all'\textit{essere} dell'ente che incontra all'interno del proprio mondo. Ontologicamente, comprendiamo chi siamo incontrando altri enti che comprendiamo non essere noi.

Il mondo è l'orizzonte in cui dobbiamo operare: questo orizzonte cambia da individuo a individuo (o forse meglio dire tra macrogruppi di individui), data la natura interpretativa dell'esistenza dell'uomo nel mondo.

L’esser-ci è accerchiato da cose a cui dare un significato utile alla realizzazione dei suoi progetti: perciò l’esistenza dell’esser-ci è caratterizzata dalla possibilità; l’uomo ha davanti a sé indefinite possibilità da realizzare, che si traducono nella possibilità di progettare. Questo libertà dell'uomo è una conseguenza del rapporto con l'\textit{essere} che abbiamo da sempre, rapporto ermeneutico, bene ricordarlo.

Adesso il problema si sposta sulla \textit{mondità} del mondo, l'\textit{essere} mondo del mondo. Non si deve procedere:

\begin{enumerate}
	\item Enumerando e descrivendo gli enti del mondo, rimanendo sul piano \textit{ontico} (relativo all'ente) e non ontologico (relativo all'\textit{essere}); qualsiasi descrizione presuppone la \textit{mondità}.
	\item Svelare l'\textit{essere} dell'ente presente nel mondo, ovvero la natura, perché così facendo presupponiamo la \textit{mondità} della natura.
\end{enumerate}

Poiché l'essere-nel-mondo è un esistenziale, un carattere costitutivo dell'esistenza umana, la mondità va cercata nell'uomo; dobbiamo condurre un'"analitica dell'esser-ci" per indagare il mondo \footnote{Nota personale: si ricade in qualche tipo di idealismo? Il soggetto proietta le sue leggi sul mondo-oggetto o addirittura pone il mondo come contrapposizione al soggetto-uomo.}.
	
Il mondo più prossimo all'esser-ci nella sua quotidianità è il mondo-ambiente: l'uomo si comprende inanzi tutto attraverso "un commercio con il mondo e con gli enti intramondani", non mediante il conoscere percettivo, ma attraverso il "prendersi cura (da non intendersi in senso morale) usante e maneggiante" dell'ente (strumento del e nel mondo); usando e maneggiando oggetti, l'essere umano comprende cosa non è, e l'esistenza viene interpretata a partire dall'ente nell'ambiente che ci circonda, che ha il suo senso d'\textit{essere} in quanto mezzo e strumento, tanto da affermare che l'\textit{utilizzabilità} è il modo d'\textit{essere} del mezzo. Importante notare che l'esistenza umana è compresa non attraverso la contemplazione della natura-oggetto (percezione), che è legata al momento presente, bensì attraverso l'interazione e l'uso degli strumenti della natura o da essa ricavati, che danno una prospettiva temporale all'\textit{essere} dell'esserci; progettando, l'uomo trova il suo significato anche nel passato e nel futuro.


Ogni mezzo dell'uomo, non è mezzo isolato, bensì appartiene alla totalità dei mezzi, con cui è in relazione: ogni mezzo è "qualcosa per ..." che rinvia ad un ulteriore "qualcosa per ...", eccetera. Il "per" contiene implicitamente sempre un rimandare di qualcosa a qualcos'altro. Ogni commercio con un mezzo sottostà alla molteplicità dei rimandi costitutivi del "per", ogni volta che uno strumento è usato rimanda ad un altro e così via attraverso le infinite relazioni di tutti gli enti del mondo tra loro: \textit{la visione ambientale preveggente} è la visione connessa a questo processo, cioè, ogni volta che l'uomo progetta, mette in relazione le cose del mondo tra loro (esempio: una lavagna assume un certo significato se è posta vicino ad una cattedra, perché la presenza della cattedra rimanda all’idea che ci troviamo in un’aula), in maniera inconscia, senza consapevolezza e tematizzazione, attraverso pre-comprensione (in pratica attraverso la visione ambientale preveggente ci muoviamo nel mondo come sempre l'abbiamo conosciuto, grazie a tradizioni, educazione, linguaggio).

Collegato a questo tema del rimando vi è poi quello fondamentale, già accennato, della pre-comprensione. Quando siamo in un atteggiamento di comprensione del mondo, questa comprensione non arriva mai dal nulla, ma parte sempre da una rete di significati già presenti in noi: questi significati derivano dalla rete di rimandi che abbiamo descritto (la visione ambientale preveggente, nel senso che l’ambiente condiziona la nostra comprensione delle cose), ma anche da rimandi più generali, come la nostra storia personale, la nostra cultura, e via dicendo. Di tutti questi strumenti di pre-comprensione il più importante è il linguaggio, in quanto è strumento di comunicazione di base con cui pensiamo e diciamo il mondo. A cascata,  il tema della pre-comprensione ci porta alla questione del \textit{circolo ermeneutico}. Da quello che abbiamo visto, conoscere il mondo non significa conoscerlo ex novo, ma interpretarlo. Questo significa che per una comprensione più adeguata dobbiamo scavare nell’interpretazione del mondo, e ogni passaggio di questa interpretazione arricchisce ulteriormente la comprensione. Si innesca quindi una circolarità di significati, un circolo ermeneutico appunto. Usando un’immagine metaforica, potremmo dire che la comprensione non è tanto un andare avanti, ma un andare indietro, ovvero ripercorrere la catena di significati che è alle spalle di una situazione attuale.

Prima di tornare al nostro cammino che ci condurrà alla semiotica, vale la pena sviluppare degli argomenti che riguardano il progettare dell'uomo come suo rapporto verso il mondo: con la dimensione della cura per l'ente, Heidegger analizza come l’esser-ci si pone in relazione alla dimensione temporale.
Qui entra in gioco il tema della differenza fra vita autentica e vita inautentica.
La vita autentica è caratterizzata dall'\textit{appartenere}, nasce da progetti che ci appartengono, che sono propriamente nostri; viceversa, la vita inautentica è legata ad una dimensione in cui non sviluppiamo progetti che sentiamo come nostri.
Per comprendere questa distinzione introduciamo il concetto di mondo del \textit{si} (\underline{si} dice, \underline{si} fa, ...). Questa dimensione è quella da cui tutti, necessariamente, passiamo. Il problema è che rischiamo di rimanerci incastrati e vivere nelle tre trappole che il mondo del si ci pone: la chiacchiera, ovvero limitarci a pensare e dire le cose che generalmente si pensano e si dicono, reputandole vere in quanto di senso comune; la curiosità, ovvero l’interessarci della vita altrui rimanendo nella superficie dell’apparenza visibile; l’equivoco, ovvero pensare che quello che emerge dalla chiacchiera e dalla curiosità rappresenti realmente la verità.
Rimanendo dentro a questa dimensione entriamo nella vita inautentica, in quanto rinunciamo a scavare la verità e finiamo per vivere una vita che non ci appartiene, una vita di conformismo. Subiamo un processo di deiezione, ovvero diventiamo una cosa fra le cose, una semplice presenza. Rinunciamo alla nostra esistenza autentica, ovvero a produrre progetti che ci appartengono.
L’alternativa alla vita inautentica nasce se ci poniamo come esseri-per-la-morte. Ovvero: prendiamo coscienza della dimensione più profonda della morte. La morte è la possibilità più paradossale, in quanto il suo giungere pone fine a tutte le altre possibilità esistenziali. Questo significa anche che la morte è l’unica possibilità necessaria, che cancella l’esistenza. Eppure, l’orizzonte della morte può essere decisivo per costruire una vita autentica: se accettiamo la necessità della morte, e quindi la necessità del nulla, accettiamo la nostra esistenza per quello che è. Invece di fuggire la morte non pensandoci, la trasformiamo in decisione anticipatrice. Questo significa che possiamo riempire il tempo di significato, possiamo trasformare il tempo in istante, per usare un’immagine di Nietsche, ovvero in un attimo denso di significato. Attraverso questa operazione recuperiamo appieno il nostro rapporto col tempo: la deiezione della vita inautentica ci spinge in una sorta di eterno presente, la decisione anticipatrice fa calare nel presente dei nostri progetti lo sguardo sul futuro.

Le esperienze tramite cui illuminiamo il mondo, i momenti in cui emerge il carattere di rimando proprio di ogni utilizzabile in quanto mezzo, avvengono quando scopriamo l'inutilizzabilità di un oggetto utilizzabile, permettendoci di andare oltre la pre-comprensione che ci fa assumere tutto per scontato. Le esperiende dove pesdiamo l'utilizzabilità di un ente sono tre:

\begin{itemize}
	\item sorpresa, quando l'oggetto non è idoneo;
	\item importunità, quando l'oggetto è mancante;
	\item impertinenza, quando l'oggetto ostacola il nostro progetto.
\end{itemize}

Con l'apparire di almeno uno dei tre imprevisti, il mondo compare ai nostri occhi, si stacca dallo sfondo "già costantemente visto sin dal principio nel corso della visione ambientale preveggente, visto in maniera non tematica (cioè pre-compreso ma non compreso). Questa emersione del mondo dovuta all'errore può permettere la tematizzazione del mondo stesso e la sua comprensione intesa come totalità di tutti i rimandi. \textbf{Ogni rimando è un segno, cioè un mezzo il cui specifico carattere di mezzo è quello di indicare}. 

Essendo ogni ente usato e interpretato (\textit{curato}) dall'uomo, ogni oggetto è un segno che rimanda ad un'interpretazione o ad un significato, che nella visione ambientale preveggente fa emergere un complesso di mezzi così da far annunciare la conformità al mondo propria dell'utilizzabile. Ogni segno annuncia "ciò che sta venendo" a cui non eravamo preparati perché indaffarati in altro (esempio: la lavagna nera ci annuncia una classe, una cattedra e una scuola), indicano dove si vive.
L'essere dell'ente utilizzabile ha la struttura del rimando, ogni oggetto ha in sé il carattere di rimandare e di essere rimandato. 

"Ogni ente ha con sé, presso qualcosa, il suo appagamento": secondo H. l'utilizzabile è caratterizzato dall'\textit{appagatività}, che è l'\textit{essere} dell'ente intramondano, ciò a cui esso è rimesso, il suo scopo di utilizzo (esempio: il martello è appagato quando martella; la lavagna quando è scritta e sta in una classe). Così come ogni utilizzabile rimanda a qualcosa, altrettanto l'appagatività è inserita in una catena: martellare serve per costruire, che serve per edificare, che serve per abitare. La catena mette capo, infine, alla \textit{totalità delle appagatività}, distinta come:

\begin{itemize}
	\item L'\textit{in-vista-di-cui}: l'esser-ci è ciò verso cui si muove la catena delle appagatività, verso cui l'utilizzabile si manifesta nella sua catena di rimandi; in altre parole tutti gli strumenti hanno la loro utilizzabilità (appagatività) stabilita dall'interpretazione che l'uomo gli attribuisce, mettendoli in relazione l'uno all'altro in base ai suoi progetti.
	\item Il \textit{presso-che}: l'appagatività è già sempre compresa nel mondo pre-compreso da sempre dall'uomo.
\end{itemize} 

Il mondo è l'orizzonte ermeneutico dell'uomo: l'esserci è sempre un'interpretazione di sé, del suo essere nel mondo. L'appagatività di ogni utilizzabile è resa possibile dall'aver sempre pre-compreso il mondo (carattere esistenziale dell'esser-ci): in altre parole, far parte di una determinata cultura e tradizione, assegna a tutto quello che ci circonda un significato tipico di quella certa cultura.

Possiamo concludere che la comprensione (interpretazione) del mondo è la mondità del mondo, che rivela la catena del rimandare, tutti i suoi rapporti di appagatività (utilizzabilità). Se per significare intendiamo il carattere di rimandare dei rapporti, la totalità dell'appagatività è la significatività del mondo.

L'esser-ci significa (rimanda) a se stesso che ha da conoscere il suo \textit{essere} e il suo poter \textit{essere} a partire dal suo \textit{essere} nel mondo: ogni uomo si distingue da un altro in base a come interpreta il mondo, a come vi si interfaccia e vi si muove. Avendo però ogni uomo una pre-comprensione del mondo (esempio: la cultura e la popolazione in cui nasciamo), esso si ritrova già in una certa struttura del mondo, espressa dalla significatività (totalità dei rapporti del significare).

L'esser-ci è sempre rinviato ad un mondo che gli viene incontro, che è strutturato come significatività (totalità dei rapporti di appagatività/ utilizzabilità). Il mondo (cui apparteniamo) porta con sè la possibilità di essere compreso da parte dell'esser-ci, trovando i significati (di tutti gli enti, le loro relazioni, il loro utilizzo), i quali, a loro volta, fondano la possibilità della parola e del linguaggio.


\end{document}