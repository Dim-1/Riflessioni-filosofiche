\documentclass[a4paper,12pt,oneside]{article}%book o scrbook, da provare; twoside per libro; b5 per libro piccolo
\usepackage[T1]{fontenc}
\usepackage[utf8]{inputenc}
\usepackage[italian]{babel}
%\usepackage{layaureo}
%\usepackage[eulerchapternumbers,beramono,pdfspacing]{classicthesis}
%\usepackage{arsclassica}

\begin{document}
	\author{Sandro Della Maggiore}
	\title{Essere e tempo per la semiotica}
	\date{Luglio 2024}
	
	\maketitle


"Essere e Tempo" (Sein und Zeit) è un’opera del 1927 del filosofo tedesco Martin Heidegger. Divisa in tre sezioni, è un’opera incompiuta, intatti l’ultima parte non è mai stata realizzata. In Essere e Tempo Heidegger affronta il problema dell’\textit{essere}, o meglio, del senso dell'\textit{essere}; questo problema è stato eluso per millenni da tutto il pensiero filosofico occidentale, per vari motivi, che lui controbatte:

\begin{enumerate}
	\item L'\textit{essere} è il concetto più generale, perciò vuoto, senza contenuti da esplicare; per H. ciò significa che l'\textit{essere}, nella sua generalità, è il concetto più oscuro, e va quindi chiarito.
	\item Non possiamo affermare "l'\textit{essere} è ..." senza usare il verbo "\textit{essere}", quindi l'\textit{essere} è indefinibile; per H. l'impossibilità di definire l'\textit{essere} indica che esso non è un'ente, quindi cosa è?
	\item \textit{Essere} è un concetto ovvio, dove già ci intendiamo, non vale la pena soffermarcisi; per H. il fatto di \textit{pre-comprendere} l'\textit{essere} richiede ancora più fortemente la comprensione del suo senso, di capire cosa sappiamo senza saperlo.
\end{enumerate}

La domanda che ci poniamo dunque è questa: "Cosa significa che l'ente \underline{è} ed \underline{è} così?

Vi è un cercato, l'\textit{essere}, o meglio il suo senso, e il cercante, noi uomini che pre-comprendiamo l'\textit{essere}, ponendo il problema di questa pre-comprensione stessa. L'\textit{essere} è ciò che determina l'ente in quanto tale: se fosse stato un ente che desse modo agli enti di esistere, sarebbe Dio o qualcosa di metafisico; H. rifiuta questa soluzione, egli vuole cercare l'\textit{essere} nell'ente, come mai è stato fatto in precedenza. Al che sorge la domanda: in quale ente va cercato l'\textit{essere}? In quello che già lo pre-comprende, nel cercante, l'uomo, che da sempre si rapporta con l'\textit{essere}.

H. designa noi cercanti con il termine di \textit{esser-ci} (\textit{das sein}, tradotto da altri interpreti anche come esser-qui): caratteristica fondamentale dell’uomo è la sua esistenza, che si realizza dentro a un certo tempo e un certo spazio. L’uomo, dunque, è tale perché esiste. Questa essenza dell'esser-ci, l'esistenza dell'uomo, è di tipo interpretativo, ermeneutico, ovvero noi come enti dobbiamo sempre comprenderci, costruirci prospettive di vita, vista l'indefinibilità dell'\textit{essere} citata precedentemente. La comprensione della propria esistenza può avvenire in due modi, che distinguano due stili di vita:

\begin{itemize}
	\item Vita inautentica, mediante pre-comprensione non concettuale dell'\textit{essere} dell'esser-ci.
	\item Vita autentica, mediante comprensione autentica dell'esistenza, resa possibile dal cammino intrapreso nell'opera di H..
\end{itemize} 
	
Nell'introduzione di "Essere e Tempo" H. puntualizza che la comprensione dell'\textit{essere} dell'esser-ci (cioè dell'esistenza) concerne co-originariamente la comprensione del mondo e la comprensione dell'\textit{essere} dell'ente (cioè le cose che ci circondano) nel mondo. L'esistenza dell'uomo è nel mondo, è relazione con il mondo, e l'esser-ci non \textit{ci-è} senza mondo. L'\textit{in-essere}, cioè l'\textit{essere}-nel-mondo, è un \textit{esistenziale} secondo H., è costitutivo dell'uomo, in quanto l'uomo è ente che sempre si comprende e interpreta come legato all'\textit{essere} dell'ente che incontra all'interno del proprio mondo. Ontologicamente, comprendiamo chi siamo incontrando altri enti che comprendiamo non essere noi.

Il mondo è l'orizzonte in cui dobbiamo operare: questo orizzonte cambia da individuo a individuo (o forse meglio dire tra macrogruppi di individui), data la natura interpretativa dell'esistenza dell'uomo nel mondo.

L’esser-ci è accerchiato da cose a cui dare un significato utile alla realizzazione dei suoi progetti: perciò l’esistenza dell’esser-ci è caratterizzata dalla possibilità; l’uomo ha davanti a sé indefinite possibilità da realizzare, che si traducono nella possibilità di progettare. Questo libertà dell'uomo è una conseguenza del rapporto con l'\textit{essere} che abbiamo da sempre, rapporto ermeneutico, bene ricordarlo.

Adesso il problema si sposta sulla \textit{mondità} del mondo, l'\textit{essere} mondo del mondo. Non si deve procedere:

\begin{enumerate}
	\item Enumerando e descrivendo gli enti del mondo, rimanendo sul piano \textit{ontico} (relativo all'ente) e non ontologico (relativo all'\textit{essere}); qualsiasi descrizione presuppone la \textit{mondità}.
	\item Svelare l'\textit{essere} dell'ente presente nel mondo, ovvero la natura, perché così facendo presupponiamo la \textit{mondità} della natura.
\end{enumerate}

Poiché l'essere-nel-mondo è un esistenziale, un carattere costitutivo dell'esistenza umana, la mondità va cercata nell'uomo; dobbiamo condurre un'"analitica dell'esser-ci" per indagare il mondo \footnote{Nota personale: si ricade in qualche tipo di idealismo? Il soggetto proietta le sue leggi sul mondo-oggetto o addirittura pone il mondo come contrapposizione al soggetto-uomo.}.
	
Il mondo più prossimo all'esser-ci nella sua quotidianità è il mondo-ambiente: l'uomo si comprende inanzi tutto attraverso "un commercio con il mondo e con gli enti intramondani", non mediante il conoscere percettivo, ma attraverso il "prendersi cura usante e maneggiante" dell'ente (strumento del e nel mondo); usando e maneggiando oggetti, l'essere umano comprende cosa non è, e l'esistenza viene interpretata a partire dall'ente nell'ambiente che ci circonda, che ha il suo senso d'\textit{essere} in quanto mezzo e strumento, tanto da affermare che l'\textit{utilizzabilità} è il modo d'\textit{essere} del mezzo. Importante notare che l'esistenza umana è compresa non attraverso la contemplazione della natura-oggetto (percezione), che è legata al momento presente, bensì attraverso l'interazione e uso degli strumenti della natura o da essa ricavati, che danno una prospettiva temporale all'\textit{essere} dell'esserci; progettando, l'uomo trova il suo significato anche nel passato e nel futuro.


Ogni mezzo dell'uomo, non è mezzo isolato, bensì appartiene alla totalità dei mezzi, con cui è in relazione: ogni mezzo è "qualcosa per ..." che rinvia ad un ulteriore "qualcosa per ...", eccetera. Il "per" contiene implicitamente sempre un rimandare di qualcosa a qualcos'altro. Ogni commercio con un mezzo sottostà alla molteplicità dei rimandi costitutivi del "per", ogni volta che uno strumento è usato rimanda ad un altro e così via attraverso le infinite relazioni di tutti gli enti del mondo tra loro: \textit{la visione ambientale preveggente} è la visione connessa a questo processo, cioè ogni volta che l'uomo progetta mette in relazione le cose del mondo tra loro, dando significato alla sua esistenza.



\end{document}