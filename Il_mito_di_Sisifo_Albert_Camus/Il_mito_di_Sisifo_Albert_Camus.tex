\documentclass[a4paper,12pt,oneside]{article}%book o scrbook, da provare; twoside per libro; b5 per libro piccolo
\usepackage[T1]{fontenc}
\usepackage[utf8]{inputenc}
\usepackage[italian]{babel}
%\usepackage{layaureo}
%\usepackage[eulerchapternumbers,beramono,pdfspacing]{classicthesis}
%\usepackage{arsclassica}

\begin{document}
	\author{Sandro Della Maggiore}
	\title{Osservazioni filosofiche su "Il mito di Sisifo" di Albert Camus}
	\date{Novembre 2024}
	
	\maketitle
	
Albert Camus pone quello che, secondo lui, è il vero problema filosofico: il tema del suicidio; ovvero se la vita vale la pena di essere vissuta	oppure no. In fondo chi si abbandona al suicidio confessa di essere stato superato dalla vita o di non averla compresa.

Nel quotidiano della nostra vita continuiamo a fare sempre gli stessi gesti per abitudine. Morire volontariamente è la presa di coscienza dell'inconsistenza di tale abitudine e la mancanza di ogni profonda ragione di vivere. Infatti quasi "tutti viviamo come se nessuno sapesse" d riguardo questa insensatezza dell'esistenza; o ancora, "viviamo facendo assegnamento sull'avvenire: «domani», «più tardi», «con l'età comprenderai». Queste incoerenze sono straordinarie, dato che, alla fine , si tratta di morire".

A un certo punto l'uomo non riesce a spiegare più il mondo in cui vive, vi si sente un estraneo: questo divorzio tra l'uomo e la sua vita è propriamente il senso dell'assurdo\footnote{Quello che Max Weber ha chiamato disincanto verso un mondo razionalizzato ma senza  senso e libertà per l'individuo.}. L'argomento dell'opera di Camus è stabilire la misura esatta nella quale il suicidio sia una risposta all'assurdo.


	
	
	
\end{document}