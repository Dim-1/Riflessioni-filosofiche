\documentclass[a4paper,12pt,oneside]{article}%book o scrbook, da provare; twoside per libro; b5 per libro piccolo
\usepackage[T1]{fontenc}
\usepackage[utf8]{inputenc}
\usepackage[italian]{babel}
%\usepackage{layaureo}
%\usepackage[eulerchapternumbers,beramono,pdfspacing]{classicthesis}
%\usepackage{arsclassica}

\begin{document}
	\author{Sandro Della Maggiore}
	\title{Osservazioni filosofiche su "Il mito di Sisifo" di Albert Camus}
	\date{Novembre 2024}
	
	\maketitle
	
Nella sua opera del 1942, Albert Camus pone quello che, secondo lui, è il vero problema filosofico: il tema del suicidio; ovvero se la vita vale la pena di essere vissuta	oppure no. In fondo chi si abbandona al suicidio confessa di essere stato superato dalla vita o di non averla compresa.

Nel quotidiano della nostra vita continuiamo a fare sempre gli stessi gesti per abitudine. Morire volontariamente è la presa di coscienza dell'inconsistenza di tale abitudine e la mancanza di ogni profonda ragione di vivere. Infatti quasi "tutti viviamo come se nessuno sapesse" d riguardo questa insensatezza dell'esistenza; o ancora, "viviamo facendo assegnamento sull'avvenire: «domani», «più tardi», «con l'età comprenderai». Queste incoerenze sono straordinarie, dato che, alla fine , si tratta di morire".

A un certo punto l'uomo non riesce a spiegare più il mondo in cui vive, vi si sente un estraneo: questo divorzio tra l'uomo e la sua vita è propriamente il senso dell'assurdo\footnote{Quello che Max Weber ha chiamato disincanto verso un mondo razionalizzato ma senza  senso e libertà per l'individuo.}. L'argomento dell'opera di Camus è stabilire la misura esatta nella quale il suicidio sia una risposta all'assurdo.

Poiché l'uomo acquisisce prima l'abitudine di vivere prima di quella di pensare, la sua vita può essere un continuo eludere l'assurdo. L'elusione perfetta è la speranza, speranza verso qualcosa che si deve "meritare", o inganno di coloro che vivono per un ideale che supera la vita, che le da un senso e nel mentre la tradisce.

Al senso dell'abisso si giunge proprio con la ragione, con il "pensiero che nega se stesso, non appena afferma quale è" la propria condizione. Nasce sempre da un confronto tra individuo e mondo, tra un'azione e il mondo che la supera. L'assurdo ha bisogno di entrambi gli elementi,perciò non è nell'uomo né nel mondo, ma nella loro comune presenza.

I mistici, Chestov, Kierkegaard accettano l'assurdo, l'illogicità. Non : "Ecco l'assurdo", ma: "Ecco Dio". Tutto lo sforzo logico del pensiero misticheggiante consiste nel mettere in chiaro l'assurdo, per poi accettarlo irrazionalmente, facendo così scaturire immensa speranza. Tutto viene sacrificato all'irrazionale, quindi scompare l'assurdo insieme ad uno dei termini di paragone che l'hanno generato, la ragione che non riesce a trovare un senso al mondo.

L'uomo assurdo invece non disprezza la ragione e ammette l'irrazionale, abbracciando tutti i dati dell'esperienza e dimostrandosi poco disposto ad affidarsi alla speranza. Per quest'uomo l'importante non è guarire, ma vivere con i propri mali; Kierkegaard invece vuole guarire e lo fa negando la ragione umana. E' un Dio quello degli esistenzialisti\footnote{Alcune volte Camus usa il termine "esistenzialista" per riferirsi alla corrente filosofica di cui fa parte, altre volte, come in questo caso, per designare quei pensatori che si sono posti domande esistenziali che hanno poi risolto con un atto di fede.} che si sostiene in virtù della negazione della ragione umana: essi compiono quello che Camus chiama un "suicidio filosofico".

Queste negazioni redentrici che negano l'assurdo cancellando uno dei termini che lo generano, possono nascere anche dall'ordine razionale: anche esso pretende l'eterno, come la religione. Il processo che tenta di eliminare l'assurdo dalla vita può avvenire abbracciando l'irrazionale, oppure ostinandosi a trovare le "ragioni ragionanti" a un mondo che, all'inizio, era immaginato senza principio direttivo. "Il filosofo astratto e il filosofo religioso partono dallo stesso smarrimento e si sostengano della stessa angoscia. Ma l'essenziale è dare una spiegazione."

Al contrario "l'assurdo è la ragione lucida, che accetta i proprio limiti", divorzio tra lo spirito che desidera e mondo che delude, nostalgia di unità. Kierkegaard sopprime la nostalgia, Husserl riunisce questo universo; ma la soluzione non è sopprimere l'assurdo mascherando l'evidenza. Bisogna sapere se si può vivere o se la logica prescrive se si debba morire.

L'uomo assurdo sente solo una cosa: la propria innocenza irreparabile, e grazie a esse tira avanti. Ciò che chiede a se stesso è solo di vivere ciò che sa, adattarsi a ciò che è, e non far intervenire nulla che non sia certo. E se niente è certo, è questa stessa una certezza.

Si ribalta la domanda iniziale: la vita sarà tanto meglio vissuta in quanto non avrà alcun senso; vivere cioè accettando pienamente un destino, dando vita all'assurdo, senza eluderlo cancellando uno dei termini, sapendolo guardare.

In questo senso, una posizione filosofica coerente è la rivolta, intesa come certezza di un destino schiacciante, meno la rassegnazione che dovrebbe accompagnarla, tramite la costante presenza dell'uomo a se stesso.

Il suicidio non è rivolta, è il suo contrario; come il
salto verso la fede, è l'accettazione del proprio
limite.
La rivolta invece da alla vita il suo valore, è
l'intelligenza alle prese con una realtà che la
supera, orgogliosamente. In questa coscienza (dell'assurdo) e in questa rivolta, l'uomo assurdo
attesta la sua sola verità, che è la sfida.
L'assurdo priva della libertà eterna ma restituisce,
esaltandola, la propria libertà d'azione: perché chi
annulla l'assurdo trovando un Dio o un mondo
necessario e spiegabile, trova un padrone che al
massimo concede una libertà che non è tale,
perché non dipendente da noi medesimi.
Privazione di speranza e di avvenire significano un
accrescimento nelle libertà dell'uomo.

L'uomo che
conosce le assurdità della vita capisce che
in realtà non era libero, ma guidato da pregiudizi, costretto da barriere create dai tentativi di dare
un senso alla vita.
L'uomo assurdo gode della libertà assoluta
rappresentata dal ritorno alla coscienza, mentre il
religioso non appartiene a se stesso, ma conosce
quella libertà che consiste nel non sentirsi
responsabili.

L'uomo assurdo vive senza una scala di
valori: per questo non vive il meglio (rispetto a
quale valore?), ma vuole vivere il più
possibile; ovvero vuole trovarsi di fronte al
mondo il più spesso possibile, fare il maggior
numero di esperienze che la vita gli permetta.
Ma l'assurdo non insegna che tutte le esperienze
sono prive di senso?
L'errore è pensare che una grande quantità di
esperienze dipenda dalle circostanze della
nostra vita (dall'esser gettati in un mondo già
costituito), mentre non dipende che da noi.
Tutto sta nell'essere coscienti: "sentire la propria
vita, la propria rivolta e la propria libertà il
più intensamente possibile, equivale a vivere il
più possibile"; "il presente e la successione
dei presenti davanti un'anima permanentemente
cosciente è l'ideale dell'uomo assurdo;
ideale suona falso, perché si tratta della terza
conseguenza del ragionamento assurdo, non di
vocazione".

Dall'assurdo si ricava:
\begin{enumerate}
	\item  la propria rivolta;
	\item  la propria libertà;
	\item  la propria passione (l'essere sempre presenti a
	se stessi).
\end{enumerate}


Per mezzo della coscienza, si trasforma in regola
di vita ciò che era un invito alla morte e si
rifiuta il suicidio, accettando l'assurdo come
necessario.

All'uomo assurdo la nostalgia per l'eterno non
è estranea, ma egli preferisce il proprio coraggio
(gli insegna a vivere secondo ciò che ha) e il proprio
ragionamento (gli fa conoscere i suoi limiti), dunque
preferisce la sua libertà a termine, la sua rivolta
senza avvenire.

La morale di uno spirito assurdo, nelle proprie
azioni, giudica che gli effetti devono essere
considerati con serenità, ed è pronto a pagare
se necessario; ovvero per lui vi possono essere
responsabili ma non colpevoli. Al più considererà
le passate esperienze come fondamento per i
suoi atti futuri.

Coloro che sono tristi hanno due ragioni per esserlo:
essi ignorano o sperano.
Don Giovanni e' un esempio di uomo assurdo,
cosciente del suo essere un seduttore comune. La
sua è un'etica della quantità, contrariamente
al santo che tende alla qualità: non colleziona
donne, altrimenti le esaurirebbe e insieme a
loro la probabilità di vita. Il tempo cammina
con lui, non si separa mai dal tempo, non vive nel
proprio passato.
L'insieme della vita di Don Giovanni sono tutte
le morti e le rinascite, il modo che egli ha di
dare e di far vivere. Il suo perciò non è egoismo;
mentre lo è il sentimento di un amore eterno
verso una sola persona, che annichilisce tutto il
resto. Inoltre egli non teme la vecchiaia, perché,
come qui uomo assurdo, ne è consapevole, conosce
i limiti. "E' estrema fine, attesa ma non desiderata,
rimane degna di disprezzo".

Anche i commedianti sono uomini assurdi: "di tutte le glorie la meno fallace è quella che
rivive", una gloria dopo l'altra, come la vita
dell'attore. Questo significa perdersi per poi
ritrovarsi.
"L'uomo è fine a se stesso. Ed è anche il suo solo
fine. Se vuole essere qualche cosa, deve esserlo in
questa vita", questo ci insegnano gli attori.

L'atteggiamento dell'uomo assurdo è anche
quello del conquistatore: gli piace superarsi. Ma il
suo destino sta di fronte al conquistatore, la morte,
che il egli sfida non per orgoglio ma
per coscienza della sua condizione senza valore.
Quindi ha pietà di sé stesso, l'unica compassione
accettabile, per la fine delle sue conquiste.

L'amante, il commediante e l'avventuriero recitano
l'assurdo, così come può farlo chiunque sa
e che nulla maschera. Sapere che essere privi di
speranza non significa disperare, vuol dire essere
saggi, come lo sono gli uomini assurdi, che vivono
di ciò che possiedono, senza speculare su ciò che
non hanno.



	
	
	
\end{document}