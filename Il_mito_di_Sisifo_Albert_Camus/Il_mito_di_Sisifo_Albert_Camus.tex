\documentclass[a4paper,12pt,oneside]{article}%book o scrbook, da provare; twoside per libro; b5 per libro piccolo
\usepackage[T1]{fontenc}
\usepackage[utf8]{inputenc}
\usepackage[italian]{babel}
%\usepackage{layaureo}
%\usepackage[eulerchapternumbers,beramono,pdfspacing]{classicthesis}
%\usepackage{arsclassica}

\begin{document}
	\author{Sandro Della Maggiore}
	\title{Osservazioni filosofiche su "Il mito di Sisifo" di Albert Camus}
	\date{Novembre 2024}
	
	\maketitle
	
Nella sua opera del 1942, Albert Camus pone quello che, secondo lui, è il vero problema filosofico: il tema del suicidio; ovvero se la vita vale la pena di essere vissuta	oppure no. In fondo chi si abbandona al suicidio confessa di essere stato superato dalla vita o di non averla compresa.

Nel quotidiano della nostra vita continuiamo a fare sempre gli stessi gesti per abitudine. Morire volontariamente è la presa di coscienza dell'inconsistenza di tale abitudine e la mancanza di ogni profonda ragione di vivere. Infatti quasi "tutti viviamo come se nessuno sapesse" d riguardo questa insensatezza dell'esistenza; o ancora, "viviamo facendo assegnamento sull'avvenire: «domani», «più tardi», «con l'età comprenderai». Queste incoerenze sono straordinarie, dato che, alla fine , si tratta di morire".

A un certo punto l'uomo non riesce a spiegare più il mondo in cui vive, vi si sente un estraneo: questo divorzio tra l'uomo e la sua vita è propriamente il senso dell'assurdo\footnote{Quello che Max Weber ha chiamato disincanto verso un mondo razionalizzato ma senza  senso e libertà per l'individuo.}. L'argomento dell'opera di Camus è stabilire la misura esatta nella quale il suicidio sia una risposta all'assurdo.

Poiché l'uomo acquisisce prima l'abitudine di vivere prima di quella di pensare, la sua vita può essere un continuo eludere l'assurdo. L'elusione perfetta è la speranza, speranza verso qualcosa che si deve "meritare", o inganno di coloro che vivono per un ideale che supera la vita, che le da un senso e nel mentre la tradisce.

Al senso dell'abisso si giunge proprio con la ragione, con il "pensiero che nega se stesso, non appena afferma quale è" la propria condizione. Nasce sempre da un confronto tra individuo e mondo, tra un'azione e il mondo che la supera. L'assurdo ha bisogno di entrambi gli elementi,perciò non è nell'uomo né nel mondo, ma nella loro comune presenza.

I mistici, Chestov, Kierkegaard accettano l'assurdo, l'illogicità. Non : "Ecco l'assurdo", ma: "Ecco Dio". Tutto lo sforzo logico del pensiero misticheggiante consiste nel mettere in chiaro l'assurdo, per poi accettarlo irrazionalmente, facendo così scaturire immensa speranza. Tutto viene sacrificato all'irrazionale, quindi scompare l'assurdo insieme ad uno dei termini di paragone che l'hanno generato, la ragione che non riesce a trovare un senso al mondo.

L'uomo assurdo invece non disprezza la ragione e ammette l'irrazionale, abbracciando tutti i dati dell'esperienza e dimostrandosi poco disposto ad affidarsi alla speranza. Per quest'uomo l'importante non è guarire, ma vivere con i propri mali; Kierkegaard invece vuole guarire e lo fa negando la ragione umana. E' un Dio quello degli esistenzialisti\footnote{Alcune volte Camus usa il termine "esistenzialista" per riferirsi alla corrente filosofica di cui fa parte, altre volte, come in questo caso, per designare quei pensatori che si sono posti domande esistenziali che hanno poi risolto con un atto di fede.} che si sostiene in virtù della negazione della ragione umana: essi compiono quello che Camus chiama un "suicidio filosofico".

Queste negazioni redentrici che negano l'assurdo cancellando uno dei termini che lo generano, possono nascere anche dall'ordine razionale: anche esso pretende l'eterno, come la religione. Il processo che tenta di eliminare l'assurdo dalla vita può avvenire abbracciando l'irrazionale, oppure ostinandosi a trovare le "ragioni ragionanti" a un mondo che, all'inizio, era immaginato senza principio direttivo. "Il filosofo astratto e il filosofo religioso partono dallo stesso smarrimento e si sostengano della stessa angoscia. Ma l'essenziale è dare una spiegazione."

Al contrario "l'assurdo è la ragione lucida, che accetta i proprio limiti", divorzio tra lo spirito che desidera e mondo che delude, nostalgia di unità. Kierkegaard sopprime la nostalgia, Husserl riunisce questo universo; ma la soluzione non è sopprimere l'assurdo mascherando l'evidenza. Bisogna sapere se si può vivere o se la logica prescrive se si debba morire.

L'uomo assurdo sente solo una cosa: la propria innocenza irreparabile, e grazie a esse tira avanti. Ciò che chiede a se stesso è solo di vivere ciò che sa, adattarsi a ciò che è, e non far intervenire nulla che non sia certo. E se niente è certo, è questa stessa una certezza.

Si ribalta la domanda iniziale: la vita sarà tanto meglio vissuta in quanto non avrà alcun senso; vivere cioè accettando pienamente un destino, dando vita all'assurdo, senza eluderlo cancellando uno dei termini, sapendolo guardare.

In questo senso, una posizione filosofica coerente è la rivolta, intesa come certezza di un destino schiacciante, meno la rassegnazione che dovrebbe accompagnarla, tramite la costante presenza dell'uomo a se stesso.




	
	
	
\end{document}