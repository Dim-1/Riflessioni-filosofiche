\documentclass[a4paper,12pt,oneside]{article}%book o scrbook, da provare; twoside per libro; b5 per libro piccolo
\usepackage[T1]{fontenc}
\usepackage[utf8]{inputenc}
\usepackage[italian]{babel}
%\usepackage{layaureo}
%\usepackage[eulerchapternumbers,beramono,pdfspacing]{classicthesis}
%\usepackage{arsclassica}

\begin{document}
	\author{Sandro Della Maggiore}
	\title{Riflessioni stoiche e sull'amore}
	\date{Settembre 2024}
	
	\maketitle


\textbf{Il principio direttivo della filosofia stoica consiste nel porsi le domande}:

\begin{itemize}
	\item \textbf{i miei pensieri sono coerenti con quello che ho scoperto e conosco di me stesso?}
	\item Il mio agire mi corrisponde?
	\item \textbf{le mie abitudini corrispondono alla mia volontà?}
\end{itemize}

Per lo stoico, rispettare ed agire secondo la propria volontà significa che tutto quanto scopro di me, deve essere accettato e integrato nelle mie abitudini, vedremo successivamente come.

\textbf{L'accettazione rimanda ad un tema caro allo stoicismo: il destino. Il destino è tutto ciò che avviene indipendentemente da noi, la somma di tutte le forze originate nel passato (anche lontanissimo) che determinano il nostro presente}, di cui noi in verità siamo una piccolissima parte.

Non avendo nessuna capacità per conoscere il tutto, possiamo costruire solamente delle interpretazioni: è perciò importante che queste siano sempre più ampie possibili, che alzino e varino il punto di vista per accrescere le prospettive, nella consapevolezza che vi sarà sempre uno \textbf{scarto tra la nostra mappa del mondo e il reale. Questa differenza tra le nostre rappresentazioni e ciò che realmente siamo, è la parte inconscia e irrazionale che mai potremo conoscere completamente}, ma che attraverso un'analisi su sé stessi potremo ridurre e rendere meno imprevedibile, così da diminuire il rischio di imprevisti.

\textbf{All'interno del destino si è liberi quando il principio direttivo (cercare di interpretare bene il reale) amplia la visione, così da trovare motivi e significati per ciò che accade}. Comprendere porta ad accettare e a comportarsi di conseguenza, a cambiare il proprio comportamento con la presa di coscienza delle nuove rappresentazioni e comprensioni.

\textbf{In tutto ciò,  Dio, per gli stoici, serve a dare un connotato positivo al destino. Il principio direttivo ha così una direzione verso un continuo miglioramento}, nella consapevolezza che, al saggio che vuole conoscere, viene migliorata la comprensione di un flusso degli eventi orientato verso un progresso positivo. Perciò la comprensione del destino è comprensione del bene, e ciò porta alla felicità.\footnote{Personalmente preferisco una visione del mondo e di Dio come quella di Spinoza, perché non ha bisogno di credere in un principio teleologico che guida verso il bene; all'interno di un mondo guidato da leggi necessarie, simile a quello stoico, Spinoza mostra il giusto comportamento etico (arrivando a conclusioni simili sempre agli stoici) senza fare ricorso a nessun principio morale ne trascendente ne immanente, bensì dimostrando le buone abitudini che ci portano alla felicità a partire dalle sole leggi necessarie che guidano la natura. }

Dopo destino e Dio, l'altra parola chiave dello stoicismo è \textbf{"natura" (physis), intesa come l'ordinatrice del cosmo e ciò da cui si deve trarre i significati delle cose. Contrapposta alla realtà della natura, abbiamo la doxa, ovvero le opinioni umane, dovute al consenso che vogliamo ricevere dagli altri, dal ritenere che l'approvazione è avere quello che tutti desiderano}. In tal modo il desiderio è una cosa che non ha mai fine.

Riformulando il principio direttivo iniziale possiamo affermare:\textbf{ quanto il mio comportamento dipende dalle mie tendenze naturali e quanto da opinioni e circostanze esterne?}

\textbf{Capire quanti fattori esterni è possibile togliere, è fondamentale per scoprire cosa realmente è mio e con cui riesco ancora a ben vivere e non solo a sopravvivere.}

Tornando alla \textbf{libertà stoica, essa si ricava in due modi}:

\begin{enumerate}
	\item \textbf{metodo della negazione della condizionabilità, cioè negare il superfluo e portare il più possibile fuori la nostra personalità e natura}. Quindi la massima libertà è rinunciare alle condizioni che condizionano la nostra natura. Il determinismo stoico  è caratterizzato da un certo grado di libertà: possiamo agire o spinti da ciò che ci determina esternamente o spinti dalla nostra natura interna. Questo grado di libertà che dobbiamo affermare in quanto uomini, è indispensabile per poter vivere in un mondo dove tutto è determinato e necessario, ma non conoscibile completamente a noi mortali.
	
	\item \textbf{Siamo inoltre liberi di ampliare il proprio quadro rappresentativo, in quanto non possiamo conoscere tutte le condizioni che determinano la realtà, per cui ci dobbiamo costruire delle  rappresentazioni (comunque delle illusioni) il più possibili corrispondenti alla natura e scegliamo dei comportamenti più prossimi al nostro intimo io e alla nostra volontà. }
\end{enumerate}

Non ha quindi senso spendere energie per evitare eventi che già sono avvenuti o che avverranno per accumulo di eventi passati. \textbf{Libertà è dire si a quello che accade, e quando la morte ci coglierà, dobbiamo essere felici, ovvero aver realizzato la nostra volontà, ciò che siamo, vivendo il presente senza preoccuparsi del futuro, perché pensando sempre a quello che faremo non viviamo il tempo attuale}. Vivere come nostro ogni attimo, fino al momento della morte, denota un'esistenza felice per lo stoicismo.

\textbf{In tutto ciò, dove sta l'amore?}

L'amore è probabilmente il principale strumento per conoscere se stessi, per autocomprendersi e così realizzarsi come esseri umani. Ricordiamo che per gli stoici la felicità passa dal conoscere se stessi il meglio possibile, così da assumere un comportamento che assecondi la propria natura.

Stiamo con una persona per lungo tempo non perché ci da gioia sensuale o sessuale (che hanno la loro importanza solo in un primo momento del rapporto di coppia), ma per altri motivi: per abitudine, triste ammetterlo, perché preferiamo stare male per tutta la vita ma sempre alla stessa maniera, non ci piace cambiare; oppure, ed è l'aspetto che qui ci interessa, per conoscere meglio la propria coscienza, perché nell'altro ci vediamo riflessi.

\textbf{Più diamo amore al nostro partner, più apprendiamo su di noi dalle sue reazioni, dal suo comportamento, dalle sue emozioni e dall'amore che a sua volta ci restituisce}. Amore è prendersi cura, comprendere sentimenti altrui che sono scaturiti a loro volta anche da reazioni generate dalle nostre azioni e dai nostri sentimenti; amare è un dono che ci ritorna indietro ancor più ricco, perché ci da comprensione del proprio io interiore e di quello del partner.

\textbf{Dare amore è forse il miglior modo per apprendere parzialmente le sfaccettature più oscure e inconsce della propria personalità: quelle che si rivelano solo ad un osservatore esterno e mai attraverso un proprio cammino interiore. Il partner può cogliere la nostra ombra in senso jungano, farcela conoscere meglio per poi accettarla; ciò avviene per osservazione di quei particolari per noi inconsci e inconoscibili, ma che si manifestano nel nostro comportamento}.

Il buon Barruch Spinoza concludeva nella sua etica un ragionamento simile, mostrando che per un uomo la cosa più bella che egli possa fare è amare il prossimo:

\begin{quotation}
	Solo in quanto gli umani vivono guidati dalla Ragione in
	tanto essi s’accordano per natura necessariamente e sempre.\footnote{Etica more geometrico demonstrata quarta parte proposizione 35.}
	
	Chi vive guidato dalla Ragione si sforza, per quanto può, di
	contraccambiare l’Odio, l’Ira, il Disprezzo altrui nei suoi riguardi con Amore, ossia con Generosità, cioè con la volontà attiva
	– dipendente solo dalla ragione – di essere utile agli altri e di
	farseli amici.\footnote{Etica more geometrico demonstrata quarta parte proposizione 46.}
\end{quotation}

Ovviamente il buon Spinoza fa seguire a questi enunciati le rispettive e inconfutabili dimostrazioni.

Giungendo a conclusione, l'amore (verso gli altri, sennò che amore è) aumenta la comprensione della propria anima, e così negli altri individui, determinando quindi comportamenti "felici" e un circolo virtuoso che porta ad un miglioramento non solo personale ma dell'intera comunità, a dimostrazione di come la filosofia stoica non sia assolutamente individualistica.







\end{document}